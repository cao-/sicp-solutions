\def\I[#1,#2]{[#1,\,#2]}%
To subtract two intervals, we can add to the first the opposite\fnote{If $X$ is an interval, with its “opposite”, denoted by $-X$, we mean the image of $X$ under the function $x\mapsto -x$. In other words, it is the set obtained by taking the opposite of each number in the interval~$X$. Notice, however, that $-X$ is not the algebraic opposite of $X$, since in general $X+(-X)$ is different from the interval $\I[0,0
]$, which is the neutral element of the addition of intervals.} of the second.  Note that the bounds of the opposite interval are the opposite of the upper bound and the opposite of the lower bound, in that order.
\begtt\scm
(define (sub-interval x y)
  (add-interval x
                (make-interval (- (upper-bound y))
                               (- (lower-bound y)))))
\endtt
