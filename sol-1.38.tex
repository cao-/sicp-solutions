Here is a procedure to compute the $k$-term finite continued fraction of~$e - 2$:
\begtt\scm 
(define (euler-cont-frac k)
  (define (n i) 1.0)
  (define (d i)
    (let ((r (remainder i 3)))
      (if (= r 2)
          (* 2.0 (+ (/ (- i r) 3) 1))
          1.0)))
  (cont-frac n d k))
\endtt
We can use it to find approximations of $e$: 
\begtt\scm
(+ 2 (euler-cont-frac 1))
;Value:3.0
(+ 2 (euler-cont-frac 2))
;Value:2.6666666666666665
(+ 2 (euler-cont-frac 3))
;Value:2.75
(+ 2 (euler-cont-frac 4))
;Value:2.7142857142857144
(+ 2 (euler-cont-frac 5))
;Value:2.71875
(+ 2 (euler-cont-frac 6))
;Value:2.717948717948718
(+ 2 (euler-cont-frac 7))
;Value:2.7183098591549295
(+ 2 (euler-cont-frac 8))
;Value:2.718279569892473
(+ 2 (euler-cont-frac 9))
;Value:2.718283582089552
(+ 2 (euler-cont-frac 10))
;Value:2.7182817182817183
(+ 2 (euler-cont-frac 11))
;Value:2.7182818352059925
(+ 2 (euler-cont-frac 12))
;Value:2.7182818229439496
(+ 2 (euler-cont-frac 13))
;Value:2.718281828735696
\endtt
