The procedure `p` is such that when called it calls again itself, so it is a never-terminating procedure.
If the interpreter uses applicative order, then when evaluating
\begtt
(test 0 (p))
\endtt
it must first evaluate `(p)`, so the procedure will never terminate;  instead, if the interpreter uses normal order, then by substitution the above expression is expanded into
\begtt
(if (= 0 0) 0 (p))
\endtt
and, because of the evaluation rule of the special form `if`, since the condition `(= 0 0)` is true, the procedure `p` is never called and the result is~$0$.
