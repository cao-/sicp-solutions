Using the substitution model, we get
\begtt\scm\typoscale[800/800]
(A 1 10)
(A 0 (A 1 9))
(A 0 (A 0 (A 1 8)))
(A 0 (A 0 (A 0 (A 1 7))))
(A 0 (A 0 (A 0 (A 0 (A 1 6)))))
(A 0 (A 0 (A 0 (A 0 (A 0 (A 1 5))))))
(A 0 (A 0 (A 0 (A 0 (A 0 (A 0 (A 1 4)))))))
(A 0 (A 0 (A 0 (A 0 (A 0 (A 0 (A 0 (A 1 3))))))))
(A 0 (A 0 (A 0 (A 0 (A 0 (A 0 (A 0 (A 0 (A 1 2)))))))))
(A 0 (A 0 (A 0 (A 0 (A 0 (A 0 (A 0 (A 0 (A 0 (A 1 1))))))))))
(A 0 (A 0 (A 0 (A 0 (A 0 (A 0 (A 0 (A 0 (A 0 2)))))))))
(A 0 (A 0 (A 0 (A 0 (A 0 (A 0 (A 0 (A 0 4))))))))
(A 0 (A 0 (A 0 (A 0 (A 0 (A 0 (A 0 8)))))))
(A 0 (A 0 (A 0 (A 0 (A 0 (A 0 16))))))
(A 0 (A 0 (A 0 (A 0 (A 0 32)))))
(A 0 (A 0 (A 0 (A 0 64))))
(A 0 (A 0 (A 0 128)))
(A 0 (A 0 256))
(A 0 512)
1024

(A 2 4)
(A 1 (A 2 3))
(A 1 (A 1 (A 2 2)))
(A 1 (A 1 (A 1 (A 2 1))))
(A 1 (A 1 (A 1 2)))
(A 1 (A 1 (A 0 (A 1 1))))
(A 1 (A 1 (A 0 2)))
(A 1 (A 1 4))
(A 1 (A 0 (A 1 3)))
(A 1 (A 0 (A 0 (A 1 2))))
(A 1 (A 0 (A 0 (A 0 (A 1 1)))))
(A 1 (A 0 (A 0 (A 0 2))))
(A 1 (A 0 (A 0 4)))
(A 1 (A 0 8))
(A 1 16)
(A 0 (A 1 15))
(A 0 (A 0 (A 1 14)))
(A 0 (A 0 (A 0 (A 1 13))))
(A 0 (A 0 (A 0 (A 0 (A 1 12)))))
(A 0 (A 0 (A 0 (A 0 (A 0 (A 1 11))))))
(A 0 (A 0 (A 0 (A 0 (A 0 (A 0 (A 1 10)))))))
(A 0 (A 0 (A 0 (A 0 (A 0 (A 0 (A 0 (A 1 9))))))))
(A 0 (A 0 (A 0 (A 0 (A 0 (A 0 (A 0 (A 0 (A 1 8)))))))))
(A 0 (A 0 (A 0 (A 0 (A 0 (A 0 (A 0 (A 0 (A 0 (A 1 7))))))))))
(A 0 (A 0 (A 0 (A 0 (A 0 (A 0 (A 0 (A 0 (A 0 (A 0 (A 1 6)))))))))))
(A 0 (A 0 (A 0 (A 0 (A 0 (A 0 (A 0 (A 0 (A 0 (A 0 (A 0 (A 1 5))))))))))))
(A 0 (A 0 (A 0 (A 0 (A 0 (A 0 (A 0 (A 0 (A 0 (A 0 (A 0 (A 0 (A 1 4)))))))))))))
(A 0 (A 0 (A 0 (A 0 (A 0 (A 0 (A 0 (A 0 (A 0 (A 0 (A 0 (A 0 (A 0 (A 1 3))))))))))))))
(A 0 (A 0 (A 0 (A 0 (A 0 (A 0 (A 0 (A 0 (A 0 (A 0 (A 0 (A 0 (A 0 (A 0 (A 1 2)))))))))))))))
(A 0 (A 0 (A 0 (A 0 (A 0 (A 0 (A 0 (A 0 (A 0 (A 0 (A 0 (A 0 (A 0 (A 0 (A 0 (A 1 1))))))))))))))))
(A 0 (A 0 (A 0 (A 0 (A 0 (A 0 (A 0 (A 0 (A 0 (A 0 (A 0 (A 0 (A 0 (A 0 (A 0 2)))))))))))))))
(A 0 (A 0 (A 0 (A 0 (A 0 (A 0 (A 0 (A 0 (A 0 (A 0 (A 0 (A 0 (A 0 (A 0 4))))))))))))))
(A 0 (A 0 (A 0 (A 0 (A 0 (A 0 (A 0 (A 0 (A 0 (A 0 (A 0 (A 0 (A 0 8)))))))))))))
(A 0 (A 0 (A 0 (A 0 (A 0 (A 0 (A 0 (A 0 (A 0 (A 0 (A 0 (A 0 16))))))))))))
(A 0 (A 0 (A 0 (A 0 (A 0 (A 0 (A 0 (A 0 (A 0 (A 0 (A 0 32)))))))))))
(A 0 (A 0 (A 0 (A 0 (A 0 (A 0 (A 0 (A 0 (A 0 (A 0 64))))))))))
(A 0 (A 0 (A 0 (A 0 (A 0 (A 0 (A 0 (A 0 (A 0 128)))))))))
(A 0 (A 0 (A 0 (A 0 (A 0 (A 0 (A 0 (A 0 256))))))))
(A 0 (A 0 (A 0 (A 0 (A 0 (A 0 (A 0 512)))))))
(A 0 (A 0 (A 0 (A 0 (A 0 (A 0 1024))))))
(A 0 (A 0 (A 0 (A 0 (A 0 2048)))))
(A 0 (A 0 (A 0 (A 0 4096))))
(A 0 (A 0 (A 0 8192)))
(A 0 (A 0 16384))
(A 0 32768)
65536

(A 3 3)
(A 2 (A 3 2))
(A 2 (A 2 (A 3 1)))
(A 2 (A 2 2))
(A 2 (A 1 (A 2 1)))
(A 2 (A 1 2))
(A 2 (A 0 (A 1 1)))
(A 2 (A 0 2))
(A 2 4)
\endtt
\hskip\ttindent\hskip1em\vdots
\begtt\scm\typoscale[800/800]
65536
\endtt

Directly from the definition of `A` we deduce that
$$
f(n) = \cases{0   & if $n=0$, \cr
              2n  & otherwise,}
$$
that is simply $f(n) = 2n$.

Looking at the expansion of `(A 1 10)` it is quite clear that $g(n) = 2^n$.  This can also be proved noticing that $g(0) = 0$, $g(1) = 2$, and for $n>1$
$$
g(n) = A(1, n) = A(0, A(1, n - 1)) = f(A(1, n - 1)) = f(\,g(n-1)) = 2g(n - 1)\hbox{.}
$$

In the same way, we prove that $h(0) = 0$, $h(1) = 2$ and $h(n) = g(h(n-1)) = 2^{h(n - 1)}$ for $n>1$.  So, to recap, 
$$
\eqalign{
    f(n) &= \underbrace{2 + 2 + \cdots + 2}_{n\;\mathbox{times}} \cr 
    g(n) &= \underbrace{2 \times 2 \times \cdots \times 2}_{n\;\mathbox{times}} \cr 
    h(n) &= \underbrace{2 {\,}^\wedge\, 2{\,}^\wedge\,  \cdots {\,}^\wedge\, 2}_{n\;\mathbox{times}}\hbox{.} \cr 
}
$$
