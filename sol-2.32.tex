The procedure that generates the set of subsets of a given set is
\begtt\scm
(define (subsets s)
  (if (null? s)
      (list nil)
      (let ((rest (subsets (cdr s))))
        (append rest (map (lambda (e) (cons (car s) e)) rest)))))
\endtt

This procedure works because if $S$ is a non-empty set and $x$ is one of its elements, then for each subset $T$ of $S$ there are two possibilities:
\begitems
* either $x$ does not belong to $T$, so that $T$ is a subset of $S\setminus\{x\}$;
* or $x$ is an element of $T$, in which case $T$ is the union between $\{x\}$ and some subset of $S\setminus\{x\}$.
\enditems
More formally,
$$
{\cal P}(S) = {\cal P}(S\setminus\{x\})\;\hbox{\typoscale[1200/]$\cupdot$}\;\bigl\{\, Y\cup\{x\} :\, Y\in{\cal P}(S\setminus\{x\}) \bigr\}\hbox{,}
$$
where ${\cal P}(S)$ and ${\cal P}(S\setminus\{x\})$ denote the sets of subsets of $S$ and $S\setminus\{x\}$ respectively.
