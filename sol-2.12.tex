We can define `make-center-percent` in terms of `make-center-width`:
\begtt\scm
(define (make-center-percent c p)
  (make-center-width c (* c (/ p 100.0))))

(define (percent i)
  (* (/ (width i) (center i)) 100.0))
\endtt
For instance, we can evaluate the range of resistance for a resistor labeled “$6.8$ ohms with $10\%$ tolerance”:
\begtt\scm
(define range (make-center-percent 6.8 10))
;Value:
(lower-bound range)
;Value:6.12
(upper-bound range)
;Value:7.4799999999999995
(center range)
;Value:6.8
(percent range)
;Value:9.999999999999996
\endtt

Alternatively, we can define `make-center-percent` directly in terms of `make-interval`; and define `percent` in terms of `lower-bound` and `upper-bound`, as follows:
\begtt\scm
(define (make-center-percent c p)
  (let ((r (/ p 100.0)))
    (make-interval (* c (- 1  r)) (* c (+ 1  r)))))

(define (percent i)
  (let ((a (lower-bound i))
        (b (upper-bound i)))
    (* (/ (- b a) (+ a b)) 100.0)))
\endtt
Now, if we perform the same evaluations as before, slightly different results get displayed, because the different expressions used lead to different truncation errors:
\begtt\scm
(define range (make-center-percent 6.8 10))
;Value:
(lower-bound range)
;Value:6.12
(upper-bound range)
;Value:7.48
(center range)
;Value:6.800000000000001
(percent range)
;Value:10.000000000000002
\endtt
