\begitems\style a
* The `cont-frac` procedure can be defined as follows:
\begtt\scm
(define (cont-frac n d k)
  (define (term-from j)
    (if (> j k)
        0
        (/ (n j)
           (+ (d j) (term-from (+ j 1))))))
  (term-from 1))
\endtt
Then we can compute approximations of $1/\phi$:
\begtt\scm
(define (one-over-phi k)
  (cont-frac (lambda (i) 1.0)
             (lambda (i) 1.0)
             k))

(one-over-phi 1)
;Value:1.0
(one-over-phi 2)
;Value:0.5
(one-over-phi 3)
;Value:0.6666666666666666
(one-over-phi 4)
;Value:0.6000000000000001
(one-over-phi 5)
;Value:0.625
(one-over-phi 6)
;Value:0.6153846153846154
(one-over-phi 7)
;Value:0.6190476190476191
(one-over-phi 8)
;Value:0.6176470588235294
(one-over-phi 9)
;Value:0.6181818181818182
(one-over-phi 10)
;Value:0.6179775280898876
(one-over-phi 11)
;Value:0.6180555555555556
(one-over-phi 12)
;Value:0.6180257510729613
(one-over-phi 13)
;Value:0.6180371352785146
\endtt
We see that a value of $k=11$ is enough to have an approximation that is accurate to $4$ decimal places.
* The above defined `cont-frac` procedure generates a recursive process.  Here is one that generates an iterative process:
\begtt\scm
(define (cont-frac n d k)
  (define (iter j result)
    (if (> j 0)
        (iter (- j 1)
              (/ (n j)
                 (+ (d j) result)))
        result))
  (iter k 0))
\endtt
\enditems
