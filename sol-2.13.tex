Let's simplify the problem by assuming that all numbers are positive.  So, let $x$ be an interval, with both lower and upper bounds positive.  Furthermore, let $c_x$ be the centre of $x$ and let $r_x$ be the ratio between the width and the centre of~$x$. With these quantities defined, the interval $x$ has the following expression:
$$
x = \I[c_x - r_x c_x, c_x + r_x c_x]\hbox{.}
$$

Now, if $y$ is another analogous interval with expression
$$
y = \I[c_y - r_y c_y, c_y + r_y c_y]\hbox{,}
$$
we can compute the product of $x$ and $y$:
$$
\eqalign{
z &= x \cdot y \cr 
  &= \I[(c_x - r_x c_x)(c_y - r_y c_y), (c_x + r_x c_x)(c_y + r_y c_y)] \cr 
  &= \I[c_x c_y - c_x c_y (r_x + r_y - r_x r_y), c_x c_y + c_x c_y (r_x + r_y - r_x r_y)]
}
$$

Under the assumption of small percentage tolerances (i.e., small values of $r_x$ and $r_y$), we can approximate the expression $r_x + r_y \pm r_x r_y$ with the simpler expression $r_x + r_y$.
With this approximation, $z$ simplifies into
$$
z\approx \I[c_x c_y - c_x c_y (r_x + r_y), c_x c_y + c_x c_y (r_x + r_y)].
$$

This means that $c_z = c_x c_y$ and also that $r_z = r_x + r_y$.  So, the approximate percentage tolerance of the product of two intervals is the sum of the tolerances of the factors.
