Let $X$ be the interval $\I[a,b]$, and let $Y$ be the interval $\I[c,d]$, for some numbers $a, b, c, d$.  Then $X+Y$ is the interval $\I[a+c, b+d]$, so its width is:
$$
\def\w{\mathop{\rm width}}
\eqalign{
\w(X + Y) &= {(b + d) - (a + c) \over 2} \cr
          &= {b - a \over 2} + {d - c \over 2} \cr
          &= \w(X) + \w(Y)\hbox{.}
}
$$
The difference $X-Y$ is the interval $\I[a - d, b - c]$, so its width is:
$$
\def\w{\mathop{\rm width}}
\eqalign{
\w(X - Y) &= {(b - c) - (a - d) \over 2} \cr
          &= {b - a \over 2} + {d - c \over 2} \cr
          &= \w(X) + \w(Y)\hbox{.}
}
$$
Overall, we see that the width of the sum or difference of two intervals is the sum of the widths of the two intervals. 

Now, let $X$ be the interval $\I[0,2]$ and $Y$ the interval $\I[1,3]$.  Both of them are of width~$1$.  If we multiply $X$ by $X$, we get the interval~$\I[0,4]$, whose width is~$2$;  instead, if we multiply $X$ by~$Y$ we get the interval~$\I[0,6]$, whose width is~$3$.  This shows that the width of the product of two intervals is {\em not} a function of the widths of the intervals being multiplied.  The same can be shown for division.
