\begitems\style a
* \removettskip\begtt\scm
(define (left-branch mobile)
  (car mobile))

(define (right-branch mobile)
  (cadr mobile))

(define (branch-length branch)
  (car branch))

(define (branch-structure branch)
  (cadr branch))
\endtt

* \removettskip\begtt\scm
(define (total-weight mobile)
  (+ (branch-weight (left-branch mobile))
     (branch-weight (right-branch mobile))))

(define (branch-weight branch)
  (if (mobile? (branch-structure branch))
      (total-weight (branch-structure branch))
      (branch-structure branch)))

(define (mobile? structure)
  (pair? structure))
\endtt

* A straightforward way to determine whether a mobile is balanced or not is to recursively compute the torque applied to its branches by making use of the above defined `total-weight` procedure: 
\begtt\scm
(define (balanced? mobile)
  (let ((left (left-branch mobile))
        (right (right-branch mobile)))
    (and (= (* (branch-length left) (branch-weight left))
            (* (branch-length right) (branch-weight right)))
          (or (not (mobile? (branch-structure left)))
              (balanced? (branch-structure left)))
          (or (not (mobile? (branch-structure right)))
              (balanced? (branch-structure right))))))
\endtt
Anyhow, this method is not efficient, because it performs a lot of redundant computations---weights of mobiles (except for the top-level one) are computed multiple times\fnote{The tree-recursive process generated by this procedure is similar to that generated by the `fib` procedure of section~1.2.2.}.

Another more efficient way to test if a mobile is balanced, is to avoid using the `total-weight` procedure, and instead let the `balanced?` procedure directly compute and return the weight of the given mobile when the mobile is balanced:\fnote{This is fine because in Scheme anything other than `#f` is treated as true, so returning the value `#t` is not strictly necessary.}
\begtt\scm
(define (balanced? mobile)
  (let ((left (left-branch mobile))
        (right (right-branch mobile)))
    (let ((left-result (if (mobile? (branch-structure left))
                           (balanced? (branch-structure left))
                           (branch-structure left)))
          (right-result (if (mobile? (branch-structure right))
                            (balanced? (branch-structure right))
                            (branch-structure right))))
      (and left-result
           right-result
           (= (* (branch-length left) left-result)
              (* (branch-length right) right-result))
           (+ left-result right-result)))))
\endtt

Furthermore, with a slight modification, we can avoid exploring the remaining mobiles as soon as we find a mobile that is not balanced:
\begtt\scm
(define (balanced? mobile)
  (let ((left (left-branch mobile))
        (right (right-branch mobile)))
    (let ((left-result (if (mobile? (branch-structure left))
                           (balanced? (branch-structure left))
                           (branch-structure left))))
      (and left-result
           (let ((right-result (if (mobile? (branch-structure right))
                                   (balanced? (branch-structure right))
                                   (branch-structure right))))
              (and right-result
                   (= (* (branch-length left) left-result)
                      (* (branch-length right) right-result))
                   (+ left-result right-result)))))))
\endtt

* In principle, when we modify the constructors of mobiles and branches, we may need to modify the corresponding selectors and also the procedure `mobile?` that we used to tell if a structure is a mobile, since they constitute the abstraction barriers that we built to interact with mobiles and branches without knowing their representation. In this particular case, however, only few changes are required:
\begtt\scm
(define (right-branch mobile)
  (cdr mobile))

(define (branch-structure branch)
  (cdr branch))
\endtt
\enditems
