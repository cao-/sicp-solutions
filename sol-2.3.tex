Let's assume for simplicity that we want to represent only rectangles with sides parallel to the orthogonal cartesian axes describing the plane they belong to.
Here are two ways we can use to represent such a rectangle:
\begitems\style n
* We can describe it simply by giving one of its diagonals (a segment).

Constructor:
\begtt\scm
(define (make-rectangle diagonal)
  diagonal)
\endtt

Some selectors:
\begtt\scm
(define (base r)
  (abs (- (x-point (start-segment r))
          (x-point (end-segment r)))))

(define (height r)
  (abs (- (y-point (start-segment r))
          (y-point (end-segment r)))))
\endtt
* Alternatively, we can describe it by giving its upper-left vertex (a point) and the lengths of its sides (positive numbers).
\smallskip

Constructor:
\begtt\scm
(define (make-rectangle upper-left-vertex base height)
  (cons upper-left-vertex
        (cons base height)))
\endtt

Some selectors:
\begtt\scm
(define (base r)
  (car (cdr r)))

(define (height r)
  (cdr (cdr r)))
\endtt
\enditems
\begtt\scm
\endtt

For both representations of a rectangle we defined the selectors `base` and `height` that give respectively the length of the horizontal and vertical sides of the rectangle.
These selectors are abstraction barriers that allow us to define the procedures for computing the perimeter and area of any rectangle, without knowing its representation:
\begtt\scm
(define (perimeter r)
  (* 2 (+ (base r)
          (height r))))

(define (area r)
  (* (base r)
     (height r)))
\endtt
